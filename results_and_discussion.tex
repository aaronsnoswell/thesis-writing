
\chapter{Results and Discussion}
\label{chap:results_and_discussion}

\hl{REWORK THIS}

\begin{itemize}

\item Show each light field captured, the depth information and the result after de-blurring, and compare with 2D deblurring methods.

\item Show the noise and Q-increase for each light field, compared with other 2D de-blurring methods.

\end{itemize}


\hl{REWORK ALL OF THIS}

\section{Limitations of thesis and results}

\begin{itemize}
\item Requires external calibration of depth map
\item Velocity information is provided a-priori
\end{itemize}


\section{Recap / discuss results}

Blah blah blah.

\section{Implications of results}

\begin{itemize}
\item It is possible to de-blur a LF image, at all scene depths
\end{itemize}


\section{Further research and or extrapolation}

\begin{itemize}

\item LF Depth Map quality drops when motion blur is present (show house sequence of depth maps). This suggests that better, blur-resistant LF depth algorithms are needed, or that the method presented here could be adapted to an iterative form (depth->deblur->depth again->deblur again).

\item LF Depth confidence map was not used at all - this could be used to improve the de-blurring process. E.g. in regions of low depth confidence, all candidate de-blurred pixels are compared in a local neighborhood and the sharpest pixels are selected.

\item Generalise to camera rotation or arbitrary camera trajectory

\item Generalise to object motion (local de-blurring vs. global deblurring)

\item Explore with other shapes of LF (e.g. more angular resolution)

\item Future Work and Extensions - Very last item in discussion

\end{itemize}




