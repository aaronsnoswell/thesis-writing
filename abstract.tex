\begin{abstract}


\begin{description}

\item[Aim:]
The aim of this research was to explore ways to restore motion-blurred Light Field (LF) images, specifically in a robotic environment.

\item[Technical Gap:]
LF cameras have a range of uses in robotics, however the performance of robotic systems can be affected by motion blur in an image.
Normal deconvolution de-blurring is limited in that it only works for images with one depth.

\item[Method:]
A well-known mathematical de-blurring technique, known as deconvolution, was adapted to work with LF images.
In this context LF images are characterised as containing x, y, colour and angular light information.

\item[Results:]
A depth-aware deconvolution technique that operates on LF images was developed.

\item[Conclusion:]
In this thesis we show how deconvolution can be applied to LF images, even when there is a range of depths within the image. The angular information available in the LF allows for depth-aware deconvolution, recovering sharp edges at all scene depths.

\end{description}

\end{abstract}